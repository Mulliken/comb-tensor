\documentclass{article}
\usepackage[letterpaper,total={5.5in, 8in}]{geometry}
\usepackage{siunitx,float}
\sisetup{load-configurations = abbreviations}
\usepackage{threeparttable,booktabs}
%opening
%\author{}

\begin{document}

%\maketitle

\section*{The System of Units}
\renewcommand{\arraystretch}{1.2}

We define a new time unit $\mathrm{T}$ such that a wave number $\tilde{\nu}$ has the same numerical value as the corresponding angular wave frequency $\omega=\tilde{\nu} 2\pi c$ ($c$ is the speed of light). 
\begin{table}[H]
	\centering
	\begin{threeparttable}
	\begin{tabular}{rll}
		\toprule
		               & SI                                 & New                                       \\
		\midrule
		         $\nu$ & \SI{1}{\per\centi\meter}           & $1\mathrm{cm^{-1}}$                       \\
		      $\omega$ & \SI{1.8836515673088531e11}{\per\s} & $1\mathrm{T}^{-1}$                        \\
		 Time Unit(SI) & \SI{1}{\s}                         & $\num{1.8836515673088531e11}\mathrm{\,T}$ \\
		Time Unit(New) & \SI{5.308837458876145e-12}{\s}     & $1\mathrm{T}$                             \\
		\bottomrule
	\end{tabular}
	\caption{The Defining Relationship: $\omega = 2\pi c \nu.$ $c =\SI{2.99792458e10}{\centi\meter\per\s}$.}
	\end{threeparttable}
\end{table}

With the time unit defined, we further define a new energy unit $\mathrm{E}$ such that Planck's constant $\hbar$ is $1\mathrm{E\,T}$.  Planck's constant in SI is \SI{1.054571817e-34}{\joule\s}. Once $\mathrm{E}$ is defined, for $\tilde{\nu}=\SI{1}{\centi\meter}$, we will have the corresponding angular frequency $\omega=1\mathrm{cm}^{-1}$ and the energy $\mathcal{E}=\hbar\omega = 1\mathrm{\,E\,T}\times 1\mathrm{\,T}^{-1}=1\mathrm{\,E}$.
\begin{table}[H]
	\centering
	\begin{threeparttable}
	\begin{tabular}{rll}
		\toprule
		                 & SI                                 & New                                           \\
		\midrule
		         $\hbar$ & \SI{1.054571817e-34}{\joule\s}     & $1\mathrm{E\cdot T}$                          \\
		   Time Unit(SI) & \SI{1}{\s}                         & $\num{1.8836515673088531e10} \mathrm{\,T}$ \\
		  Time Unit(New) & \SI{5.308837458876145e-12}{\s}     & $1\mathrm{T}$                                 \\
		 Energy Unit(SI) & \SI{1}{\joule}                     & $\num{5.0341165706272096e22} \mathrm{\,E}$ \\
		Energy Unit(New) & \SI{1.986445855931795e-23}{\joule} & $1 \mathrm{E}$                                \\
		\bottomrule
	\end{tabular}
	\caption{The Defining Relationship: $\hbar = \SI{1.054571817e-34}{\joule\s}  =1\mathrm{ET}$.}
	\end{threeparttable}
\end{table}

With the two units defined, we calculate the value of Boltzmann's constant in this system of units. 
\begin{table}[H]
	\centering
	\begin{threeparttable}
	\begin{tabular}{rll}
		\toprule
		                 & SI                                   & New                                           \\
		\midrule
		 Energy Unit(SI) & \SI{1}{\joule}                       & $\num{5.0341165706272096e22} \mathrm{\,E}$ \\
		Energy Unit(New) & \SI{1.986445855931795e-23}{\joule}   & $1 \mathrm{E}$                                \\
		  $k_\mathrm{B}$ & \SI{1.380649e-23}{\joule\per\kelvin} & $\num{0.6950348009119888} \mathrm{\,E\,K^{-1}}$         \\
		\bottomrule
	\end{tabular}
	\caption{The Defining Relationship: $k_{\mathrm{B}} = \SI{1.380649e-23}{\joule\per\kelvin}  =\SI{1.380649e-23}{\joule\per\kelvin}  \times \num{15.0341165706272096e22} \frac{\mathrm{E}}{\mathrm{J}} = \num{0.6950348009119888} \mathrm{E\,K^{-1}}$.}
	\end{threeparttable}
\end{table}

\end{document}
