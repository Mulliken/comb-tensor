%!TEX TS-program = pdflatex

\documentclass{article}
\usepackage[letterpaper,total={5.5in, 8in}]{geometry}
\usepackage{siunitx,float}
\sisetup{load-configurations = abbreviations}
\usepackage{threeparttable,booktabs}
%opening
%\author{}

\begin{document}

%\maketitle

\section*{The System of Units}

\renewcommand{\arraystretch}{1.2}

We define a new time unit $\mathrm{T}$ such that a wave number $\tilde{\nu}$ has the same numerical value as the corresponding angular wave frequency $\omega=\tilde{\nu} 2\pi c$ ($c$ is the speed of light). 
\begin{table}[H]
	\centering
	\begin{threeparttable}
	\begin{tabular}{rll}
		\toprule
		               & SI                                 & New                                       \\
		\midrule
		         $\tilde{\nu}$ & \SI{1}{\per\centi\meter}           & $1\mathrm{cm^{-1}}$                       \\
		      $\omega$ & $1.8836515673088531\times 10^{11}$ \SI{}{\per\s} & $1\mathrm{T}^{-1}$                        \\
		 Time Unit(SI) & \SI{1}{\s}                         & ${1.8836515673088531\times 10^{11}}\mathrm{\,T}$ \\
		Time Unit(New) & $5.308837458876145\times 10^{-12}$\SI{}{\s}     & $1\mathrm{T}$                             \\
		\bottomrule
	\end{tabular}
	\caption{The Defining Relationship: $\omega = 2\pi c \nu.$ $c =\SI{2.99792458e10}{\centi\meter\per\s}$.}
	\end{threeparttable}
\end{table}

With the time unit defined, we further define a new energy unit $\mathrm{E}$ such that Planck's constant $\hbar$ is $1\mathrm{E\,T}$.  Planck's constant in SI is \SI{1.054571817e-34}{\joule\s}. Once $\mathrm{E}$ is defined, for $\tilde{\nu}=\SI{1}{\centi\meter}$, we will have the corresponding angular frequency $\omega=1\mathrm{cm}^{-1}$ and the energy $\mathcal{E}=\hbar\omega = 1\mathrm{\,E\,T}\times 1\mathrm{\,T}^{-1}=1\mathrm{\,E}$.
\begin{table}[H]
	\centering
	\begin{threeparttable}
	\begin{tabular}{rll}
		\toprule
		                 & SI                                 & New                                           \\
		\midrule
		         $\hbar$ & $1.054571817\times 10^{-34}$\SI{}{\joule\s}     & $1\mathrm{E\cdot T}$                          \\
		   Time Unit(SI) & \SI{1}{\s}                         & $1.8836515673088531\times10^{10} \mathrm{\,T}$ \\
		  Time Unit(New) & \SI{5.308837458876145e-12}{\s}     & $1\mathrm{T}$                                 \\
		 Energy Unit(SI) & \SI{1}{\joule}                     & $5.0341165706272096\times 10^{22} \mathrm{\,E}$ \\
		Energy Unit(New) & $1.986445855931795\times10^{-23}$\SI{}{\joule} & $1 \mathrm{E}$                                \\
		\bottomrule
	\end{tabular}
	\caption{The Defining Relationship: $\hbar = 1.054571817\times10^{-34}\SI{}{\joule\s}  =1\mathrm{ET}$.}
	\end{threeparttable}
\end{table}

With the two units defined, we calculate the value of Boltzmann's constant in this system of units. 
\begin{table}[H]
	\centering
	\begin{threeparttable}
	\begin{tabular}{rll}
		\toprule
		                 & SI                                   & New                                           \\
		\midrule
		 Energy Unit(SI) & \SI{1}{\joule}                       & $5.0341165706272096\times10^{22} \mathrm{\,E}$ \\
		Energy Unit(New) & $1.986445855931795\times10^{-23}$\SI{}{\joule}   & $1 \mathrm{E}$                                \\
		  $k_\mathrm{B}$ & $1.380649\times10^{23}$\SI{}{\joule\per\kelvin} & $0.6950348009119888 \mathrm{\,E\,K^{-1}}$         \\
		\bottomrule
	\end{tabular}
	\caption{The Defining Relationship: $k_{\mathrm{B}} = \SI{1.380649e-23}{\joule\per\kelvin}  =\SI{1.380649e-23}{\joule\per\kelvin}  \times \num{15.0341165706272096e22} \frac{\mathrm{E}}{\mathrm{J}} = \num{0.6950348009119888} \mathrm{E\,K^{-1}}$.}
	\end{threeparttable}
\end{table}

Now we list the necessary unit conversion coefficients from atomic units to this new set of units
\begin{table}[H]
	\centering
	\begin{threeparttable}
		\begin{tabular}{rll}
			\toprule
			& a.u.                                   & New                                           \\
			\midrule
			Mass & 1\, a.u.                       & $1.627096855727954\times10^{11} \mathrm{\,M}$ \\
			Mass & $6.145915631756325\times10^{-12}$\, a.u.   & $1\,\mathrm{M}$                                \\
			Length& 1\, a.u.                      & $5.29177210903\times 10^{-9}$ \SI{}{\centi\meter} \\
			Energy& 1\, a.u.                      & $2.1947463068\times10^5$ E \\
			\bottomrule
		\end{tabular}
		\caption{The Defining Relationship: .}
	\end{threeparttable}
\end{table}

\end{document}
